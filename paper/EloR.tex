\documentclass[sigconf,screen]{acmart}
\pdfoutput=1

% %%%%%%%%%%%%%%%%%%%%%%%%
% %Hack to remove some irrelevant copyright info.
% \settopmatter{printacmref=false} % Removes citation information below abstract
% \renewcommand\footnotetextcopyrightpermission[1]{} % removes footnote with conference information in first column
% \pagestyle{plain} % removes running headers

% \setlength{\headheight}{27pt} 
% \setlength{\footskip}{14pt}

% \makeatletter
% \renewcommand\@formatdoi[1]{\ignorespaces}
% \makeatother
% %%%%%%%%%%%%%%%%%%%%%%%%

\usepackage{natbib}
\usepackage{algorithm}
\usepackage[noend]{algorithmic}
\usepackage{amsmath}
\usepackage{geometry}
\usepackage{verbatim}
\usepackage{tabularx}
\usepackage{graphicx}
\usepackage{amsfonts}
\usepackage{cleveref}
\usepackage{multirow}
\usepackage{microtype}
\usepackage{enumitem}
\usepackage{xcolor}
\usepackage{hyperref}
%\usepackage[
% subtle
 %moderate
%]{savetrees}
%\usepackage{blindtext}
\hypersetup{
    colorlinks=true,
    linkcolor=blue,
    filecolor=magenta,      
    urlcolor=cyan,
    citecolor=orange,
}
\urlstyle{same}
\usepackage{flushend}

%\usepackage{biblatex}
%\DeclareMathSizes{10}{9}{6}{5}
\newtheorem*{theorem*}{Theorem}
\newtheorem{theorem}{Theorem}[section]
\newtheorem{corollary}[theorem]{Corollary}
\newtheorem{lemma}[theorem]{Lemma}
\newtheorem{proposition}{Proposition}[section]
\newtheorem{claim}{Claim}[section]
\newtheorem{definition}{Definition}[section]
\newtheorem{remark}{Remark}[section]
\newtheorem{observation}{Observation}[section]
\newtheorem{condition}{Condition}[section]

\DeclareMathOperator*{\argmax}{arg\,max}
\DeclareMathOperator*{\argmin}{arg\,min}

\newcommand{\INPUT}{\item[{\bf Input:}]}
\newcommand{\OUTPUT}{\item[{\bf Output:}]}

\newcommand{\partdiff}[2]{\frac{\partial {#1}}{\partial {#2}}}
\newcommand{\secdiff}[2]{\frac{\partial^2 {#1}}{\partial {#2}^2}}
\newcommand{\mixdiff}[3]{\frac{\partial^2 {#1}}{{\partial {#2}}{\partial {#3}}}}
\newcommand{\iprod}[2]{\left\langle {#1}, 
{#2} \right \rangle}
\newcommand{\etal}{et al.\ }
\newcommand{\ceil}[1]{\lceil #1 \rceil}
\newcommand{\floor}[1]{\lfloor #1 \rfloor}
\newcommand{\bigfloor}[1]{\left\lfloor #1 \right\rfloor}
\newcommand{\bigceil}[1]{\left\lceil #1 \right\rceil}
\newcommand{\eps}{\varepsilon}
\newcommand{\E}{\mathbf{E}}
\newcommand{\Var}{\mathbf{Var}}
\newcommand{\poly}{\mathrm{poly}}
\newcommand{\rank}{\mathrm{rank}}
\newcommand{\tr}[1]{\mathrm{Tr}\left( #1 \right)}

\newcommand{\cA}{\mathcal{A}}
\newcommand{\cI}{\mathcal{I}}
\newcommand{\cM}{\mathcal{M}}
\newcommand{\cF}{\mathcal{F}}
\newcommand{\cD}{\mathcal{D}}
\newcommand{\cP}{\mathcal{P}}
\newcommand{\cE}{{\mathcal{E}}}
\newcommand{\cX}{\mathcal{X}}
\newcommand{\cH}{{\mathcal{H}}}
\newcommand{\cR}{\mathcal{R}}
\newcommand{\cB}{\mathcal{B}}
\newcommand{\cQ}{\mathcal{Q}}
\newcommand{\cZ}{\mathcal{Z}}
\newcommand{\cS}{\mathcal{S}}
\newcommand{\RR}{{\mathbb R}}
\newcommand{\ZZ}{{\mathbb Z}}
\newcommand{\NN}{{\mathbb N}}
\newcommand{\id}[1]{\mathbbm{1}_{#1}}

\newcommand{\vx}{\mathbf{x}}
\newcommand{\vy}{\mathbf{y}}
\newcommand{\xdot}{\dot \vx}
\newcommand{\ydot}{\dot \vy}
\newcommand{\dt}{\mathrm{d}t}
\newcommand{\dx}{\mathrm{d}x}
\newcommand{\dF}{\mathrm{d}F}
\newcommand{\ddp}{\frac{\mathrm{d}}{\mathrm{d}p}}

%\newcommand{\qed}{\hfill{\rule{2mm}{2mm}}}

\def\b1{{\bf 1}}
\def\be{{\bf e}}
\def\ba{{\bf a}}
\def\bg{{\bf g}}
\def\bX{{\bf X}}
\def\bb{{\bf b}}
\def\bc{{\bf c}}
\def\bx{{\bf x}}
\def\by{{\bf y}}
\def\bv{{\bf v}}
\def\bV{{\bf V}}
\def\bw{{\bf w}}
\def\bz{{\bf z}}
\def\tx{\tilde{\bf x}}

\newcommand{\paul}[1]{{\color{blue} [[PAUL: #1]]}}
\newcommand{\aram}[1]{{\color{violet} [[ARAM: #1]]}}
\DeclareMathOperator\sech{sech}
\DeclareMathOperator\erf{erf}

\title{Elo-MMR: A Rating System for Massive Multiplayer Competitions}

%\title{An Elo-like System for Massive Multiplayer Competitions}

\author{Aram Ebtekar}
\affiliation{%
  %unaffiliated: \institution{University of British Columbia}
  \city{Vancouver}
  \state{BC}
  \country{Canada}
}
\email{aramebtech@gmail.com}

\author{Paul Liu}
\affiliation{%
  \institution{Stanford University}
  \city{Stanford}
  \state{CA}
  \country{USA}
}
\email{paul.liu@stanford.edu}

\copyrightyear{2021}
\acmYear{2021}
\setcopyright{iw3c2w3}
\acmConference[WWW '21]{Proceedings of the Web Conference 2021}{April 19--23, 2021}{Ljubljana, Slovenia}
\acmBooktitle{Proceedings of the Web Conference 2021 (WWW '21), April 19--23, 2021, Ljubljana, Slovenia}
\acmPrice{}
\acmDOI{10.1145/3442381.3450091}
\acmISBN{978-1-4503-8312-7/21/04}

\settopmatter{printacmref=true}

\begin{CCSXML}
<ccs2012>
<concept>
<concept_id>10002951.10003317.10003338.10003343</concept_id>
<concept_desc>Information systems~Learning to rank</concept_desc>
<concept_significance>500</concept_significance>
</concept>
<concept>
<concept_id>10010147.10010257.10010293.10010300</concept_id>
<concept_desc>Computing methodologies~Learning in probabilistic graphical models</concept_desc>
<concept_significance>500</concept_significance>
</concept>
</ccs2012>
\end{CCSXML}

\ccsdesc[500]{Information systems~Learning to rank}
\ccsdesc[500]{Computing me-thodologies~Learning in probabilistic graphical models}

\keywords{rating system, skill estimation, mechanism design, competition, bayesian inference, robust, incentive-compatible, elo, glicko, trueskill}

\begin{document}

%\everypar{\looseness=-1}
\linepenalty=200

\begin{abstract}
Skill estimation mechanisms, colloquially known as rating systems, play an important role in competitive sports and games. They provide a measure of player skill, which incentivizes competitive performances and enables balanced match-ups. In this paper, we present a novel Bayesian rating system for contests with many participants. It is widely applicable to competition formats with discrete ranked matches, such as online programming competitions, obstacle courses races, and video games. The system's simplicity allows us to prove theoretical bounds on its robustness and runtime. In addition, we show that it is \emph{incentive-compatible}: a player who seeks to maximize their rating will never want to underperform. Experimentally, the rating system surpasses existing systems in prediction accuracy, and computes faster than existing systems by up to an order of magnitude.
\end{abstract}

\maketitle
% \pagestyle{plain}


\section{Introduction}

This paper describes the Elo-R rating system, designed for use in programming competitions. The ``R" in the name may stand for ``Ranked", as it operates on game outcomes given as a rank-ordered list of players; or it may stand for ``Robust", as its ratings are robust to outlier performances, like someone having a bad day in which their Internet connection died.

Competitions, in the form of sports, games and examinations, have been with us since antiquity. Whether for entertainment, training, or selection for a particular role, contestants and spectators alike are interested in estimating the relative skills of contestants. Skills are easiest to compare when there is a standard quantifiable objective, such as a score on a standardized test or a completion time in a race.  However, most team sports, as well as games such as chess, have no such measure. Instead, a good player is simply one who can frequently win against other good players.

The Elo rating system assigns a quantitative measure of skill to each player. For example, an average player may be rated 1500, while a top player may exceed 2500. The scale is arbitrary, but can be used to rank players relative to one another. The odds of any one player winning against another may be estimated from the difference in their ratings. The famous Elo system, and variants such as Glicko \cite{glicko}, provide useful formulas for updating ratings of players who compete 1v1 against one another, resulting in a winner and a loser. These algorithms have some nice properties: they are fairly simple, fast, and only modify the ratings of the two participating players.

Now let's consider a general setting in which a typical contest has much more than two participants. An arbitrary number of players compete simultaneously at a task. Rather than producing just a winner and a loser, the contest ranks its participants 1st place, 2nd, 3rd, and so on. This description fits popular programming competition websites such as Codeforces \cite{Codeforces} and TopCoder \cite{TopCoder}, each of which has tens of thousands of rated members from across the globe. Each publishes its own rating system, but without much theoretical justification to accompany the derivations.

We build Elo-R upon a more rigorous probabilistic model, mirroring the Bayesian development of the Glicko system. In doing so, we inherit its nice properties in practice, resolving known issues with the Codeforces and TopCoder systems. Compared with these systems, it achieves faster convergence and robustness against unusual performances. Issues specific to Codeforces than we improve upon are the overall spread of ratings, inter-division boundary artifacts, and inflation (TODO: cite, and quantify this with tests). An issue specific to TopCoder that we eliminate completely is non-monotonicity: simply put, there are cases in which improving a member's past performance would actually decrease their TopCoder rating, and vice-versa \cite{forivsektheoretical}.

Furthermore, Elo-R retains simplicity and efficiency on par with the other systems. To demonstrate this, I provide a very efficient parallel implementation that can process the entire history of rated competitions hosted by Codeforces on a modest quad-core laptop within 30 minutes. It is implemented entirely within the safe subset of Rust using the Rayon crate; hence, the Rust compiler verifies that it contains no data races.

This paper is organized as follows: in section 2, we develop a Bayesian model for the competitions. Rating updates are phrased as a latent skill estimation problem, which is naturally divided into two phases. Sections 3 and 4 describe each of these phases in turn, and supplement the derivations with some intuitive interpretations. Then in section 5, we discuss some ways to model uncertainty, in a manner analogous to the Glicko system. In section 6, we discuss some properties of the Elo-R system in comparison with the Codeforces and TopCoder systems. Finally, section 7 presents the conclusions of this work.

\section{A Bayesian Model for Massive Competitions}
\label{sec:bayes_model}
% The method is best-suited for competitions in which the most pertinent information is round rankings. Note that, in the programming contest setting, this means we discard specific information about player scores, which are difficult to model and depend heavily on the specifics of the problem set. This method would be ill-suited to model, for instance, track races, where a runner's absolute time is more informative than relative rankings. On the other hand, it may be very well-suited to obstacle-course races, if each round consists of novel obstacles that make the absolute times hard to interpret.

We now describe the setting formally, denoting random variables by capital letters. A series of competitive \textbf{rounds}, indexed by $t=1,2,3,\ldots$, take place sequentially in time. Each round has a set of participating \textbf{players} $\cP_t$, which may in general overlap between rounds. A player's \textbf{skill} is likely to change with time, so we represent the skill of player $i$ at time $t$ by a real random variable $S_{i,t}$.

In round $t$, each player $i\in \cP_t$ competes at some \textbf{performance} level $P_{i,t}$, typically close to their current skill $S_{i,t}$. The deviations $\{P_{i,t}-S_{i,t}\}_{i\in\cP_t}$ are assumed to be i.i.d. and independent of $\{S_{i,t}\}_{i\in\cP_t}$.

Performances are not observed directly; instead, a ranking gives the relative order among all performances $\{P_{i,t}\}_{i\in\cP_t}$. In particular, ties are modelled to occur when performances are exactly equal, a zero-probability event when their distributions are continuous.\footnote{
%If $e$ contains ties, then $(E_t = e)$ has probability zero in our model. 
The relevant limiting procedure is to treat performances within $\epsilon$-width buckets as ties, and letting $\epsilon\rightarrow 0$. This technicality appears in the proof of \Cref{thm:uniq-max}.} This ranking constitutes the observational \textbf{evidence} $E_t$ for our Bayesian updates. The rating system seeks to estimate the skill $S_{i,t}$ of every player at the present time $t$, given the historical round rankings $E_{\le t} := \{ E_1,\ldots,E_t \}$.

We overload the notation $\Pr$ for both probabilities and probability densities: the latter interpretation applies to zero-probability events, such as in $\Pr(S_{i,t} = s)$. We also use colons as shorthand for collections of variables differing only in a subscript: for instance, $P_{:,t}:=\{P_{i,t}\}_{i\in\cP_t}$. The joint distribution described by our Bayesian model factorizes as follows:
\begin{align}
    &\Pr(S_{:,:}, P_{:,:}, E_:) \label{eq:model}
    \\&= \prod_i \Pr(S_{i,0})
    \prod_{i,t} \Pr(S_{i,t}\mid S_{i,t-1})
    \prod_{i,t} \Pr(P_{i,t}\mid S_{i,t})
    \prod_t \Pr(E_t\mid P_{:,t}), \nonumber
\end{align}
\vspace{-1.5em}
\begin{align*}
    \text{where } \Pr(S_{i,0}) &\text{ is the initial skill prior,}
    \\\Pr(S_{i,t}\mid S_{i,t-1}) &\text{ is the skill evolution model (\Cref{sec:skill-drift}),}
    \\\Pr(P_{i,t}\mid S_{i,t}) &\text{ is the performance model, and}
    \\\Pr(E_t\mid P_{:,t}) &\text{ is the evidence model.}
\end{align*}
For the first three factors, we will specify log-concave distributions (see \Cref{def:log-concave}). The evidence model, on the other hand, is a deterministic indicator. It equals one when $E_t$ is consistent with the relative ordering among $\{P_{i,t}\}_{i\in\cP_t}$, and zero otherwise.

Finally, our model assumes that the number of participants $|\cP_t|$ is large. %(in practice, in the tens to thousands).
The main idea behind our algorithm is that, in sufficiently massive competitions, from the evidence $E_t$ we can infer very precise estimates for $\{P_{i,t}\}_{i\in\cP_t}$. Hence, we can treat these performances as if they were observed directly.
%\begin{theorem}
%Consider a round $t$ with player $i$ having performance $P_{i,t}$, and a set of participants $\cP_t$ drawn from the same distribution as player $i$. Suppose are prior believe on $P_{i,t}$ is continuous and has positive density at its true value. Then with probability 1, in the limit $|\cP_t|\rightarrow \infty$, the posterior belief on $P_{i, t}$ conditioned on $E_t$ concentrates around its true value. Furthermore,
%\[\lim_{|\cP_t|\rightarrow\infty} \Pr(S_{i,t}=s \mid P_{i,<t},\,E_t) = \Pr(S_{i,t} = s \mid P_{i, \leq t}). \]
%\end{theorem}
%\begin{proof}

That is, suppose we have the skill prior at round $t$:
\begin{equation}
\label{eq:pi-s}
\pi_{i,t}(s) := \Pr(S_{i,t} = s \mid P_{i,<t}).
\end{equation}

Now, we observe $E_t$. By \Cref{eq:model}, it is conditionally independent of $S_{i,t}$, given $P_{i,\le t}$. By the law of total probability,
\begin{align*}
&\Pr(S_{i,t}=s \mid P_{i,<t},\,E_t)
\\&= \int \Pr(S_{i,t}=s \mid P_{i,<t},\,P_{i,t}=p) \Pr(P_{i,t}=p \mid P_{i,<t},\,E_t) \, \mathrm{d}p
\\&\rightarrow \Pr(S_{i,t}=s \mid P_{i,\le t}) \quad\text{almost surely as }|\mathcal P_t|\rightarrow\infty.
\end{align*}
The integral is intractable in general, since the performance posterior $\Pr(P_{i,t}=p \mid P_{i,<t},\,E_t)$ depends not only on player $i$, but also on our belief regarding the skills of all $j\in\cP_t$. However, in the limit of infinite participants, Doob's consistency theorem~\cite{F63} implies that it concentrates at the true value $P_{i,t}$. Since our posteriors are continuous, the convergence holds for all $s$ simultaneously.
%\end{proof}

Indeed, we don't even need the full evidence $E_t$. Let $E^L_{i,t} = \{j\in\cP:P_{j,t}>P_{i,t}\}$ be the set of players against whom $i$ lost, and $E^W_{i,t} = \{j\in\cP:P_{j,t}<P_{i,t}\}$ be the set of players against whom $i$ won. That is, we only see who wins, draws, and loses against $i$. $P_{i,t}$ remains identifiable using only $(E^L_{i,t}, E^W_{i,t})$, which will be more convenient for our purposes.

Passing to the limit in which $P_{i,\le t}$ is identified serves to justify several common simplifications made by total-order rating systems. Firstly, since $P_{i,\le t}$ is a sufficient statistic for predicting $S_{i,t}$, it may be said that $(E^L_{i,\le t}, E^W_{i,\le t})$ are ``almost sufficient'' for $S_{i,t}$: any additional information, such as from domain-specific scoring systems, becomes redundant for the purposes of skill estimation. Secondly, conditioned on $P_{:,\le t}$, the posterior skills $S_{:,t}$ are independent of one another. As a result, there are no inter-player correlations to model, and a player's posterior is unaffected by rounds in which they are not a participant. Finally, if we've truly identified $P_{i,t}$, then rounds later than $t$ should not prompt revisions in our estimate for $P_{i,t}$. This obviates the need for expensive whole-history update procedures~\cite{DHMG07,WHR}, for the purposes of present skill estimation.\footnote{As opposed to \emph{historical} skill estimation, which is concerned with $P(S_{i,t} \mid P_{i,\le t'})$ for $t'>t$. Whole-history methods can take advantage of future information.}

Finally, a word on the rate of convergence. Suppose we want our estimate to be within $\epsilon$ of $P_{i,t}$, with probability at least $1-\delta$. By asymptotic normality of the posterior~\cite{F63}, it suffices to have $O(\frac 1{\epsilon^2}\log \frac 1\delta)$ participants.

When the initial prior, performance model, and evolution model are all Gaussian, treating $P_{i,t}$ as certain is the \emph{only} simplifying approximation we will make; that is, in the limit $|\cP_t|\rightarrow\infty$, our method performs \emph{exact} inference on \Cref{eq:model}. In the following sections, we focus some attention on generalizing the performance model to non-Gaussian log-concave families, parametrized by location and scale. We will use the logistic distribution as a running example and see that it induces robustness; however, our framework is agnostic to the specific distributions used.%For non-Gaussian performance models, we will make a few additional approximations, but we resist the temptation to approximate the posteriors by something compact. 

The prior \textbf{rating} $\mu^\pi_{i,t}$ and posterior rating $\mu_{i,t}$ of player $i$ at round $t$ should be statistics that summarize the player's prior and posterior skill distribution, respectively. We'll use the mode: thus, $\mu_{i,t}$ is the maximum a posteriori (MAP) estimate, obtained by setting $s$ to maximize the posterior $\Pr(S_{i,t}=s \mid P_{i,\le t})$. By Bayes' rule,
\begin{align}
\label{eq:new-obj}
\mu_{i,t}^\pi &:= \argmax_{s} \pi_{i,t}(s), \nonumber
\\\mu_{i,t} &:= \argmax_{s} \pi_{i,t}(s) \Pr(P_{i,t} \mid S_{i,t}=s).
\end{align}

This objective suggests a two-phase algorithm to update each player $i\in\cP_t$ in response to the results of round $t$. In phase one, we estimate $P_{i,t}$ from $(E^L_{i,t}, E^W_{i,t})$. By Doob's consistency theorem, our estimate is extremely precise when $|\cP_t|$ is large, so we assume it to be exact. In phase two, we update our posterior for $S_{i,t}$ and the rating $\mu_{i,t}$ according to \Cref{eq:new-obj}.


\section{Skill estimation in two phases}
\label{sec:main-alg}
    
\section{Performance Estimation}

Since our assumptions imply the posterior on $p_i$ to be narrowly concentrated, the MAP estimate is as good as any. Thus, we seek to maximize

\[f(p_i\mid e) \propto f(p_i) \Pr(e\mid p_i)\]

In the prior distribution, $p_i = s_i + (p_i-s_i)$ is a sum of two independent random variables. $s_i$ has the prior distribution, whose mode is the pre-round rating $r'_i$; let $\sigma_i^2$ be its variance. The other term $p_i-s_i$ is a zero-centered logistic random variable with variance $\gamma_i^2$.

The sum of two normal random variables is another normal variable whose mean is the sum of the component means. Inspired by this fact, we make another approximation: although neither $s_i$ nor $p_i-s_i$ are normal, we treat their sum $p_i$ as a logistic random variable centered at $r'_i$. By independence of its two components, $p_i$ has variance $\delta_i^2 = \gamma_i^2 + \sigma_i^2$.

We'll adopt the convention that any symbol that represents a standard deviation is multiplied by $\frac{\sqrt{12}}{\pi}\approx 1.1$ when drawn with an upper-bar. For example, $\bar\delta_i = \frac{\sqrt{12}}{\pi} \delta_i$. In terms of $\bar\delta_i$, the logistic p.d.f. takes a convenient form:

\[
f(p_i)
\approx \frac { 2e^{2(p_i-r'_i)/\bar\delta_i} } { \bar\delta_i\left( 1 + e^{2(p_i-r'_i)/\bar\delta_i} \right)^2}
= \frac { 1 } { 2\bar\delta_i \cosh^2\frac{p_i-r'_i}{\bar\delta_i} }
\]

Since we are working in the limit of a large number of players, we have sufficient evidence to determine $p_i$ even after ignoring relations like $j \succ k$ which don't include $i$. Thus, we can imagine $e$ to be the evidence consisting solely of, for each $j\ne i$, whether $j \succ i$ or $j \prec i$. Taking $p_i$ as fixed in the following probability expressions, using the logistic cumulative density function (c.d.f.), and ignoring constants of proportionality that depend on $e$ but not on $p_i$:

\begin{align*}
\Pr(e\mid p_i)
&= \prod_{j \succ i} \Pr(p_j > p_i) \prod_{j \prec i} \Pr(p_j < p_i)
\\&\approx \prod_{j \succ i} \frac {1} {1 + e^{2(p_i-r'_j)/\bar\delta_j}} \prod_{j \prec i} \frac {e^{2(p_i-r'_j)/\bar\delta_j}} {1 + e^{2(p_i-r'_j)/\bar\delta_j}}
\\&\propto \frac {e^{2p_i\sum_{j\prec i}1/\bar\delta_j}} {\prod_{j\neq i} 1 + e^{2(p_i-r'_j)/\bar\delta_j}}
\end{align*}

Taking logarithms, there exist constants $C$ and $C'$ such that

\begin{align*}
C + \ln f(p_i\mid e)
&= C' + \ln f(p_i) + \ln \Pr(e\mid p_i)
\\&\approx \frac{2}{\bar\delta_i} (p_i-r'_i) - 2\ln\left(1 + e^{2(p_i-r'_i)/\bar\delta_i} \right) + 2p_i\sum_{j\prec i} \frac{1}{\bar\delta_j} - \sum_{j\neq i} \ln\left(1 + e^{2(p_i-r'_j)/\bar\delta_j}\right)
\end{align*}

To maximize this expression, differentiate it w.r.t. $p_i$ and set the result to zero:
\begin{align*}
0 &= \frac{2}{\bar\delta_i}\left(1 - \frac {2e^{2(p_i-r'_i)/\bar\delta_i}} {1 + e^{2(p_i-r'_i)/\bar\delta_i}} \right)
+ \sum_{j\neq i}\frac{2}{\bar\delta_j}\left(\mathbb{I}(j\prec i) - \frac {e^{2(p_i-r'_j)/\bar\delta_j}} {1 + e^{2(p_i-r'_j)/\bar\delta_j}} \right)
\\&= \sum_{j\preceq i}\frac{2}{\bar\delta_j}\left(\frac {1} {1 + e^{2(p_i-r'_j)/\bar\delta_j}} \right)
- \sum_{j\succeq i}\frac{2}{\bar\delta_j}\left(\frac {e^{2(p_i-r'_j)/\bar\delta_j}} {1 + e^{2(p_i-r'_j)/\bar\delta_j}} \right)
\\&= -\left( \sum_{j\preceq i}\frac{1}{\bar\delta_j}\left( \tanh\frac {p_i - r'_j} {\bar\delta_j} - 1 \right)
+ \sum_{j\succeq i}\frac{1}{\bar\delta_j}\left( \tanh\frac {p_i - r'_j} {\bar\delta_j} + 1 \right) \right)
\end{align*}

By monotonicity of the $\tanh$ function, we can solve for $p_i$ by a simple binary search. If faster convergence is desired, Newton's method can be used: since $\frac{d}{dx}\tanh(x) = 1 - \tanh^2(x)$, a Newton step is very efficient, requiring no more hyperbolic function evaluations than a binary search step.

On the second line, the terms in parentheses can be thought of as a measure of surprise at the outcomes between $i$ and $j$: they are the probability of the outcomes opposite to what actually occurred, when the performance of player $i$ is fixed to $p_i$. In addition to the actual outcomes which come from $e$, the prior hallucinates two regularizing outcomes: one in which player $i$ wins against itself, and one in which player $i$ loses against itself. This regularization prevents the first- and last-place players from achieving $p_i = \pm\infty$. By choosing $p_i$ such that the sum equals zero, we are effectively saying that the total surprise from wins should equal the total surprise from losses.

Here's another intuitive interpretation: if the $\delta_j$s are all equal, this amounts to finding the performance level $p_i$ at which one's expected rank would equal player $i$'s actual rank, after accounting for the regularizing clones of player $i$. With unequal $\delta_j$s, this interpretation can be extended to a weighted ranking in which player $j$ counts $\frac{1}{\delta_j}$ times.

Now we see an easy way to handle ties: simply treat them as half a win and half a loss. Since the above expression subtracts 1 for each win and adds 1 for each loss, averaging the two yields an offset of 0 for ties. It's a hack, but an elegant one!
    \subsection{Belief update}
\label{sec:belief}

Having estimated $P_{i,t}$ in the first phase, the second phase is rather simple. Ignoring normalizing constants, \Cref{eq:new-obj} tells us that the pdf of the skill posterior can be obtained as the pointwise product of the pdfs of the skill prior and the performance model. When both factors are differentiable and log-concave, then so is their product. Its maximum is the new rating $\mu_{i,t}$; let's see how to compute it for the same two specializations of our model.

\paragraph{Gaussian performance model}
When the skill prior and performance model are Gaussian with known means and variances, multiplying their pdfs yields another known Gaussian. Hence, the posterior is compactly represented by its mean $\mu_{i,t}$, which coincides with the MAP and rating; and its variance $\sigma_{i,t}^2$, which is our \textbf{uncertainty} regarding the player's skill.

\paragraph{Logistic performance model}
When the performance model is non-Gaussian, the multiplication does not simplify so easily. By \Cref{eq:new-obj}, each round contributes an additional factor to the belief distribution. In general, we allow it to consist of a collection of simple log-concave factors, one for each round in which player $i$ has participated. Denote the participation history by
\[\cH_{i,t} := \{k\in\{1,\ldots,t\}:i\in\mathcal P_k\}.\]

Since each player can be considered in isolation, we'll omit the subscript $i$. Specializing to the logistic setting, each $k\in\cH_t$ contributes a logistic factor to the posterior, with mean $p_k$ and variance $\beta_k^2$. We still use a Gaussian initial prior, with mean and variance denoted by $p_0$ and $\beta_0^2$, respectively. Postponing the discussion of skill evolution to \Cref{sec:skill-drift}, for the moment we assume that $S_k=S_0$ for all $k$. The posterior pdf, up to normalization, is then
\begin{align}
&\pi_0(s) \prod_{k\in\cH_t} \Pr(P_k=p_k \mid S_k=s) \nonumber
\\&\propto \exp\left( -\frac{(s-p_0)^2}{2\beta_0^2} \right) \label{eq:posterior}
\prod_{k\in\cH_t} \sech^{2}\left( \frac\pi{\sqrt{12}} \frac{s-p_k} {\beta_k} \right).
\end{align}

Maximizing the posterior density amounts to minimizing its negative logarithm. Up to a constant offset, this is given by
\begin{align*}
L(s) &:= L_2\left(\frac{s-p_0}{\beta_0}\right)
+ \sum_{k\in\cH_t} L_R\left(\frac{s-p_k}{\beta_k}\right),
\\\text{where }L_2(x) &:= \frac 12 x^2\text{ and }
L_R(x) := 2\ln\left(\cosh \frac{\pi x}{\sqrt{12}}\right).
\end{align*}
\begin{equation}
\label{eq:loss}
\text{Thus, }L'(s) = \frac{s-p_0}{\beta_0^2} + \sum_{k\in\cH_t} \frac{\pi}{\beta_k\sqrt{3}} \tanh \frac{(s-p_k)\pi}{\beta_k\sqrt{12}}.
\end{equation}

$L'$ is continuous and strictly increasing in $s$, so its zero is unique: it is the MAP $\mu_t$. Similar to what we did in the first phase, we can solve for $\mu_t$ with binary search or other root-solving methods.

We pause to make an important observation. From \Cref{eq:loss}, the rating carries a rather intuitive interpretation: Gaussian factors in $L$ become $L_2$ penalty terms, whereas logistic factors take on a more interesting form as $L_R$ terms. From \Cref{fig:l2-lr-plot}, we see that the $L_R$ term behaves quadratically near the origin, but linearly at the extremities, effectively interpolating between $L_2$ and $L_1$ over a scale of magnitude $\beta_k$ 
%\aram{cite literature to justify this claim, and the next one? It would take more space to derive it ourselves}.

It is well-known that minimizing a sum of $L_2$ terms pushes the argument towards a weighted mean, while minimizing a sum of $L_1$ terms pushes the argument towards a weighted median. With $L_R$ terms, the net effect is that $\mu_t$ acts like a robust average of the historical performances $p_k$. Specifically, one can check that
\[\mu_t = \frac{\sum_k w_k p_k}{\sum_k w_k}, \text{ where } w_0 := \frac{1}{\beta_0^2} \text{ and }\]
\begin{equation}
\label{eq:average}
w_k := \frac{\pi}{(\mu_t-p_k)\beta_k\sqrt{3}}\tanh\frac{(\mu_t-p_k)\pi}{\beta_k\sqrt{12}} \text{ for }k\in\cH_t.
\end{equation}

$w_k$ is close to $1/\beta_k^2$ for typical performances, but can be up to $\pi^2/6$ times more as $|\mu_t-p_k| \rightarrow 0$, or vanish as $|\mu_t-p_k| \rightarrow\infty$. This feature is due to the thicker tails of the logistic distribution, as compared to the Gaussian, resulting in an algorithm that resists drastic rating changes in the presence of a few unusually good or bad performances. We'll formally state this \emph{robustness} property in \Cref{thm:robust}.

%Empirically, contest performances have indeed been seen to have thick tails, more like the logistic than the Gaussian (TODO citation).

\paragraph{Estimating skill uncertainty} While there is no easy way to compute the variance of a posterior in the form of \Cref{eq:posterior}, it will be useful to have some estimate $\sigma_t^2$ of uncertainty. There is a simple formula in the case where all factors are Gaussian. Since moment-matched logistic and normal distributions are relatively close (c.f. \Cref{fig:l2-lr-plot}), we apply the same formula:
\begin{equation}
\label{eq:variance}
\frac{1}{\sigma_t^2} := \sum_{k\in\{0\}\cup\cH_t}\frac{1}{\beta_k^2}.
\end{equation}
    %\subsection{Initial Skill Distributions}

%\paul{Does our system preserve the mean rating of the population?}

%\aram{No it doesn't. Should this section be here? The newbie prior is not the only thing subject to hyperparameter search.}

%In our two-phase algorithm, skill distributions can be updated from round to round via \Cref{alg:update}. However, new players to the rating system must be initialized with an initial skill distribution. In practice, any initial distribution converges after the player participates in a few rounds. Thus, one approach is to seed the rating system with an initial pool of players from which an initial skill distribution can be estimated after a few rounds of play. Another approach is a simple hyper-parameter search over the initial skill distribution that optimizes for the prediction accuracy. In our experiments, we take the latter approach.
\section{Skill evolution over time}
\label{sec:skill-drift}
%This is an important component in applications, as players often train and improve between rounds.\footnote{This skill drift is modelled by almost all of the systems described in the introduction.} The time-varying skill is typically modelled by Gaussian noise, as we describe in \Cref{sec:skill-drift}.

Factors such as training and resting will change a player's skill over time. If we model skill as a static variable, our system will eventually grow so confident in its estimate that it will refuse to admit substantial changes. To remedy this, we introduce a skill evolution model, so that in general $S_t \neq S_{t'}$ for $t \neq t'$. Now rather than simply being equal to the previous round's posterior, the skill prior at round $t$ is given by
\begin{equation}
\label{eq:drift}
\pi_t(s) = \int \Pr(S_t = s \mid S_{t-1} = x) \Pr(S_{t-1} = x \mid P_{<t}) \,\dx.
\end{equation}

The factors in the integrand are the skill evolution model and the previous round's posterior, respectively. Following other Bayesian rating systems (e.g., Glicko, Glicko-2, and TrueSkill~\cite{G99, G12, HMG06}), we model the skill diffusions $S_t-S_{t-1}$ as independent zero-mean Gaussians. That is, $\Pr(S_t \mid S_{t-1}=x)$ is a Gaussian with mean $x$ and some variance $\gamma_t^2$. The Glicko system sets $\gamma_t^2$ proportionally to the time elapsed since the last update, corresponding to a continuous Brownian motion. Codeforces and Topcoder simply set $\gamma_t$ to a constant when a player participates, and zero otherwise, corresponding to changes that are in proportion to how often the player competes. Now we are ready to complete the two specializations of our rating system.

\paragraph{Gaussian performance model}
If both the prior and performance distributions at round $t-1$ are Gaussian, then the posterior is also Gaussian. Adding an independent Gaussian diffusion to our posterior on $S_{t-1}$ yields a Gaussian prior on $S_t$. By induction, the skill belief distribution forever remains Gaussian. This Gaussian specialization of the Elo-MMR framework lacks the R for robustness (see \Cref{thm:robust}), so we call it Elo-MM$\chi$.

\paragraph{Logistic performance model}
After a player's first contest round, the posterior in \Cref{eq:posterior} becomes non-Gaussian, rendering the integral in \Cref{eq:drift} intractable.

A very simple approach would be to replace the full posterior in \Cref{eq:posterior} by a Gaussian approximation with mean $\mu_t$ (equal to the posterior MAP) and variance $\sigma_t^2$ (given by \Cref{eq:variance}). As in the previous case, applying diffusions in the Gaussian setting is a simple matter of adding means and variances.

With this approximation, no memory is kept of the individual performances $P_t$. Priors are simply Gaussian, while posterior densities are the product of two factors: the Gaussian prior, and a logistic factor corresponding to the latest performance. To ensure robustness (see \Cref{sec:robust}), $\mu_t$ is computed as the argmax of this posterior \emph{before} replacement by its Gaussian approximation. We call the rating system that takes this approach Elo-MMR($\infty$).

As the name implies, it turns out to be a special case of Elo-MMR($\rho$). In the general setting with $\rho \in [0,\infty)$, we keep the full posterior from \Cref{eq:posterior}. Since we cannot tractably compute the effect of a Gaussian diffusion, we seek a heuristic derivation of the next round's prior, retaining a form similar to \Cref{eq:posterior} while satisfying many of the same properties as the intended diffusion.

\subsection{Desirable properties of a ``pseudodiffusion''}
\label{sec:desirable-props}
We begin by listing some properties that our skill evolution algorithm, henceforth called a ``pseudodiffusion'', should satisfy. The first two properties are natural:
\begin{itemize}[leftmargin=*]
\item \emph{Incentive-compatibility.} First and foremost, the pseudodiffusion must not break the incentive-compatibility of our rating system. That is, a rating-maximizing player should never be motivated to lose on purpose. (\Cref{thm:mono}).
\item \emph{Rating preservation.} The pseudodiffusion must not alter the $\argmax$ of the belief density. That is, the rating of a player should not change: $\mu^\pi_t = \mu_{t-1}$.
\end{itemize}
In addition, we borrow four properties of Gaussian diffusions:
\begin{itemize}[leftmargin=*]
\item \emph{Correct magnitude.} Pseudodiffusion with parameter $\gamma^2$ must increase the skill uncertainty, as measured by \Cref{eq:variance}, by $\gamma^2$.
\item \emph{Composability.} Two pseudodiffusions applied in sequence, first with parameter $\gamma_1^2$ and then with $\gamma_2^2$, must have the same effect as a single pseudodiffusion with parameter $\gamma_1^2 + \gamma_2^2$.
\item \emph{Zero diffusion.} In the limit as $\gamma \rightarrow 0$, the effect of pseudodiffusion must vanish, i.e., not alter the belief distribution.
\item \emph{Zero uncertainty.} In the limit as $\sigma_{t-1}\rightarrow 0$ (i.e., when the previous rating $\mu_{t-1}$ is a perfect estimate of $S_{t-1}$), our belief on $S_t$ must become Gaussian with mean $\mu_{t-1}$ and variance $\gamma^2$. Finer-grained information regarding the prior history $P_{\le t}$ must be erased.
\end{itemize}
In particular, Elo-MMR($\infty$) fails the \emph{zero diffusion} property because it simplifies the belief distribution, even when $\gamma=0$. In the proof of \Cref{thm:diffuse-prop}, we'll see that Elo-MMR($0$) fails the \emph{zero uncertainty} property. Thus, it is in fact necessary to have $\rho$ strictly positive and finite. In \Cref{sec:robust}, we'll come to interpret $\rho$ as a kind of inverse momentum.

\subsection{A heuristic pseudodiffusion algorithm}
\label{sec:pseudodiffusion}
Each factor in the posterior (see \Cref{eq:posterior}) has a parameter $\beta_k$. Define a factor's \textbf{weight} to be $w_k := 1/\beta_k^2$, which by \Cref{eq:variance} contributes to the \textbf{total weight} $\sum_k w_k=1/\sigma_t^2$. Here, unlike in \Cref{eq:average}, $w_k$ does not depend on $|\mu_t-p_k|$.

The approximation step of Elo-MMR($\infty$) replaces all the logistic factors by a single Gaussian whose variance is chosen to ensure that the total weight is preserved. In addition, its mean is chosen to preserve the player's rating, given by the unique zero of \Cref{eq:loss}. Finally, the diffusion step of Elo-MMR($\infty$) increases the Gaussian's variance, and hence the player's skill uncertainty, by $\gamma_t^2$; this corresponds to a decay in the weight.

To generalize the idea, we interleave the two steps in a continuous manner. The approximation step becomes a \textbf{transfer step}: rather than replace the logistic factors outright, we take away the same fraction from each of their weights, and \emph{place the sum of removed weights onto a new Gaussian factor}. In order for this operation to preserve ratings, the new factor must be centered at $\mu_{t-1}$. Since Gaussian pdfs compose, the prior Gaussian factor can be combined with the new one. The diffusion step becomes a \textbf{decay step}, reducing each factor's weight by the same fraction, chosen such that the overall uncertainty is increased by $\gamma_t^2$.

To make the idea precise, we generalize the posterior from \Cref{eq:posterior} with fractional \textbf{multiplicities} $\omega_k$; the $k$'th factor is raised to the power $\omega_k$. As a result, \Cref{eq:loss,eq:variance} become:
%initially set to $1$ for each $k\in\{0\}\cup\cH_t$. 
%The $k$'th factor is raised to the power $\omega_k$; in \Cref{eq:loss,eq:variance}, the corresponding term is multiplied by $\omega_k$.

\begin{align}
\label{eq:multiplicities}
L'(s) &= \frac{\omega_0(s-p_0)}{\beta_0^2} + \sum_{k\in\cH_t} \frac{\omega_k\pi}{\beta_k\sqrt{3}} \tanh \frac{(s-p_k)\pi}{\beta_k\sqrt{12}},\nonumber
\\\frac{1}{\sigma_t^2} &:= \sum_{k\in\{0\}\cup\cH_t}w_k,\quad\text{where }w_k := \frac{\omega_k}{\beta_k^2}.
\end{align}

For $\rho\in [0,\infty]$, the Elo-MMR($\rho$) algorithm continuously and simultaneously performs transfer and decay, with transfer proceeding at $\rho$ times the rate of decay. Holding $\beta_k$ fixed, changes to $\omega_k$ can be described in terms of changes to $w_k$:
\begin{align*}
\dot w_0 &= -r(t)w_0 + \rho r(t) \sum_{k\in\cH_t} w_k,
\\\dot w_k &= -(1+\rho)r(t)w_k \quad\text{for }k\in\cH_t,
\end{align*}
where the arbitrary decay rate $r(t)$ can be eliminated by a change of variable $\mathrm{d}\tau = r(t)\dt$. After some time $\Delta\tau$, the total weight will have decayed by a factor $\kappa := e^{-\Delta\tau}$, resulting in the new weights:
\begin{align*}
w_0^{new} &= \kappa w_0 + \left(\kappa-\kappa^{1+\rho}\right)\sum_{k\in\cH_t} w_k,
\\w_k^{new} &= \kappa^{1+\rho}w_k \quad\text{for }k\in\cH_t.
\end{align*}
In order for the uncertainty to increase from $\sigma_{t-1}^2$ to $\sigma_{t-1}^2+\gamma_t^2$, we must solve $\kappa/\sigma_{t-1}^2 = 1/(\sigma_{t-1}^2+\gamma_t^2)$ for the decay factor:
\[\kappa_t = \left(1 + \frac{\gamma_t^2}{\sigma_{t-1}^2}\right)^{-1}.\]

\setlength{\floatsep}{0pt}
\setlength{\textfloatsep}{1em}
\begin{algorithm}[t]
\caption{Elo-MMR($\rho,\beta, \gamma$)}
\label{alg:main}
\begin{algorithmic}
\FORALL{rounds $t$}
\FORALL{players $i\in\mathcal P_t$ in parallel}
\IF{$i$ has never competed before}
\STATE {$\mu_i, \sigma_i \gets \mu_{newcomer}, \sigma_{newcomer}$}
\STATE {$p_i, w_i \gets [\mu_i], [1/\sigma_i^2]$}
\ENDIF
%\STATE $\gamma \gets$ systemspecified()
\STATE diffuse($i,\gamma,\rho$)
\STATE $\mu^\pi_i, \delta_i \gets \mu_i,\sqrt{\sigma_i^2 + \beta^2}$
%\STATE Make $\mu^\pi_i,\delta_i$ accessible to all threads in the next loop
\ENDFOR
%\STATE $E \gets$ getevidence()
\FORALL{$i\in\mathcal P_t$ in parallel}
\STATE update($i,E_t,\beta$)
\ENDFOR
\ENDFOR
\end{algorithmic}
\end{algorithm}
\begin{algorithm}[t]
\caption{diffuse($i,\gamma,\rho$)}
\label{alg:diffuse}
\begin{algorithmic}
\STATE $\kappa \gets (1+\gamma^2/\sigma_i^2)^{-1}$
\STATE $w_G, w_L \gets \kappa^\rho w_{i,0}, (1-\kappa^\rho) \sum_{k\geq 0} w_{i,k}$
\STATE $p_{i,0} \gets (w_G p_{i,0} + w_L \mu_i) / (w_G+w_L)$
\STATE $w_{i,0} \gets \kappa (w_G+w_L)$
\FORALL{$k > 0$}
\STATE $w_{i,k} \gets \kappa^{1+\rho}w_{i,k}$
\ENDFOR
\STATE $\sigma_i \gets \sigma_i / \sqrt\kappa$
\end{algorithmic}
\end{algorithm}
\begin{algorithm}[t]
\caption{update($i,E,\beta$)}
\label{alg:update}
\begin{algorithmic}
\STATE $p \gets \mathrm{zero}_x\left(\sum_{j\preceq i}\frac{1}{\delta_j}\left( \tanh\frac {x - \mu^\pi_j} {2\bar\delta_j} - 1 \right) + \sum_{j\succeq i}\frac{1}{\delta_j}\left( \tanh\frac {x - \mu^\pi_j} {2\bar\delta_j} + 1 \right)\right)$
\STATE $p_i$.push($p$)
\STATE $w_i$.push($1/\beta^2$)
\STATE $\mu_i \gets \mathrm{zero}_x\left(w_{i,0}(x-p_{i,0}) + \sum_{k>0} \frac{w_{i,k}\beta^2}{\bar\beta} \tanh \frac {x-p_{i,k}} {2\bar\beta}\right)$
\end{algorithmic}
\end{algorithm}

\Cref{alg:main} details the full Elo-MMR($\rho$) rating system. The main loop runs whenever a round of competition takes place. First, new players are initialized with a Gaussian prior. Then, changes in player skill are modeled by \Cref{alg:diffuse}; note how the Gaussian prior is updated into a weighted combination with the newly created factor. Given the round rankings $E_t$, the first phase of \Cref{alg:update} solves an equation to estimate $P_t$. Finally, the second phase solves another equation for the rating $\mu_t$. 

The hyperparameters $\rho,\beta,\gamma$ are domain-dependent, and can be set by standard hyperparameter search techniques. For convenience, we assume $\beta$ and $\gamma$ are fixed and use the shorthand $\bar\beta_k := \frac{\sqrt{3}}{\pi} \beta_k$. Whereas our exposition used global round indices, here a subscript $k$ corresponds to the $k$'th round in player $i$'s participation history.

\begin{theorem}
\label{thm:diffuse-prop}
\Cref{alg:diffuse} with $\rho\in(0,\infty)$ meets all of the properties listed in \Cref{sec:desirable-props}.
\end{theorem}

\begin{proof}
We go through each of the six properties in order.
\begin{itemize}[leftmargin=*]
    \item \emph{Incentive-compatibility.} This property will be stated in \Cref{thm:mono}. To ensure that its proof carries through, the relevant facts to note here are that the pseudodiffusion algorithm ignores the performances $p_k$, and centers the transferred Gaussian weight at the rating $\mu_{t-1}$, which is trivially monotonic in $\mu_{t-1}$.
    \item \emph{Rating preservation.} Recall that the rating is the unique zero of $L'$ in \Cref{eq:multiplicities}. To see that this zero is preserved, note that the decay and transfer operations multiply $L'$ by constants ($\kappa_t$ and $\kappa_t^\rho$, respectively), before adding the new Gaussian term, whose contribution to $L'$ is zero at its center.
    \item \emph{Correct magnitude.} Follows from our derivation for $\kappa_t$.
    \item \emph{Composability.} Follows from \emph{correct magnitude} and the fact that every pseudodiffusion follows the same differential equations.
    \item \emph{Zero diffusion.} As $\gamma\rightarrow 0$, $\kappa_t\rightarrow 1$. Provided that $\rho<\infty$, we also have $\kappa_t^\rho\rightarrow 1$. Hence, for all $k\in\{0\}\cup\cH_t$, $w_k^{new} \rightarrow w_k$.
    \item \emph{Zero uncertainty.} As $\sigma_{t-1}\rightarrow 0$, $\kappa_t\rightarrow 0$. The total weight decays from $1/\sigma_{t-1}^2$ to $\gamma^2$. Provided that $\rho > 0$, we also have $\kappa_t^\rho\rightarrow 0$, so these weights transfer in their entirety, leaving behind a Gaussian with mean $\mu_{t-1}$, variance $\gamma^2$, and no additional history. \qedhere
\end{itemize}
\end{proof}



\section{Properties}

First, we discuss some properties that Elo-R has in common with the published systems of Codeforces and TopCoder, as well as the classics Elo and Glicko. All of these systems propagate belief changes forward in time, never backward. This approach is simple, efficient, and has the benefit of never retroactively changing ratings from the past, nor the ratings of player who are not actively competing. Elo-R and Glicko converge to the right results a bit faster than the others, by including an uncertainty parameter that starts high for new players.

The two-phase approach of Elo-R is a bit unique, in that it's not memoryless (unless the memory is set to merge all the way down to a length of 1). Rating changes depend not only on the current rating and uncertainty, but on a list of recent performance values. Thus, we cannot make the same guarantees as Codeforces \cite{Codeforces}. This is the price of a robust system: it's impossible to identify and eliminate outliers if we don't remember their values! Nevertheless, we have some analoguous guarantees:
\begin{itemize}
\item The rating is a monotonic function of the list of past performances. Thus, unlike on Topcoder \cite{forivsektheoretical}, a situation will never arise where we would wish to have scored less.
\item If you swap the order of two performances in the list, the rating goes up if the better performance moves forward in time, and down if the better performance moves backward. This follows from the fact that newer performances have smaller uncertainty, since uncertainties don't depend on the value of any performance.
\item The performance $p_i$ is measured in the same units as rating and has the property that, within a given contest, a higher ranking contestant always has higher $p_i$ than a lower ranking one. In case of ties, the contestant with higher rating $r_i$ also has (slightly) higher $p_i$.
\end{itemize}

The code and ratings of real Codeforces members as computed by Elo-R are available at https://github.com/EbTech/EloR. Original Codeforces ratings are at http://codeforces.com/ratings. One striking difference is massive inflation in the Codeforces system. Gennady Korotkevich, best known by his competitive programming handle ``tourist", has been the reigning world champion for years. Toward the end of 2011, his rating reached a new ceiling of about 2700 according to both systems. However, as of this writing, his rating on Elo-R has increased by about 300 additional points, while on Codeforces it increased by almost 900. To get a sense of the magnitude of this change, 900 points is the difference between an average member and a Grandmaster! Indeed, most of the variance in the Codeforces system is concentrated at the top, with much smaller rating differences between beginner and intermediate members. This is caused by certain ad hoc elements of the system that are not founded on any rigorous model.

This paper will not evaluate the predictive accuracy of Elo-R; my experience with it suggests it does better than the other local methods listed above, but it is possible to do better with global methods. It's difficult to do a fair evaluation because it's not clear what exactly some of these models are trying to predict, besides the qualitative assertion that players with higher ratings should win more often. For example, Elo-R might be judged on the log-likelihood of observed match results, according to its own belief model. However, the joint likelihood is difficult to compute, and many of the other systems lack a corresponding belief model. One reasonable approach would be to approximate the likelihood as we did when estimating $p_i$, and use cross-validation to optimize the parameters of each rating system according to this criterion. Such evaluations will remain open to future investigation. Instead, let's focus on a unique feature of Elo-R that's absent in the Codeforces system, and arguably in TopCoder as well.

\subsection{Analysis of Monotonicity}

This is a draft in progress. Constant factor noising satisfies monotonicity, but it appears the fancier method can break monotonicity in some extreme cases. The purpose of this analysis is to understand when that happens and if/how that should be fixed.

Let $f$ be a function of the form:

\[f(x) = \sum_{k=1}^n \frac{1}{\tau_k} \tanh \frac {x-\mu_k} {\tau_k}\]

Then $f$ is a strictly increasing function of $x$. Let $r$ be the unique point at which $f(r) = 0$. Similarly, define

\[g(x) = \sum_{k=1}^m \frac{1}{\sigma_k} \tanh \frac {x-\nu_k} {\sigma_k}\]

such that $g(x) \ge f(x)$ for all $x$. Then $g$ too is strictly increasing: let $s$ be the unique point at which $g(s) = 0$. We cannot have $s > r$, as that would imply $0 = g(s) \ge f(s) > f(r) = 0$. Hence, $s \le r$. Furthermore, $\sum_k \frac{1}{\tau_k} = \sum_k \frac{1}{\sigma_k}$ because

\[\sum_k \frac{1}{\tau_k}
= \lim_{x\rightarrow\infty} f(x)
\le \lim_{x\rightarrow\infty} g(x)
= \sum_k \frac{1}{\sigma_k}
= -\lim_{x\rightarrow -\infty} g(x)
\le -\lim_{x\rightarrow -\infty} f(x)
= \sum_k \frac{1}{\tau_k}\]

Let $\eta$ be the noising operation:

\[\eta[f](x) = \sum_k \frac{1}{\tau'_k} \tanh \frac {x-\mu_k} {\tau'_k}\]
\[\eta[g](x) = \sum_k \frac{1}{\sigma'_k} \tanh \frac {x-\nu_k} {\sigma'_k}\]

\[ \frac{\sum_k 1 / \tau'^2_k}{\sum_k 1 / \tau^2_k} = \frac{\sum_k 1 / \sigma'^2_k}{\sum_k 1 / \sigma^2_k} \]

\[\frac{1}{\tau'_k} \tanh \frac {r-\mu_k} {\tau'_k}
= \frac{1}{\kappa^2\tau_k} \tanh \frac {r-\mu_k} {\tau_k}\]

\[\frac{1}{\sigma'_k} \tanh \frac {s-\nu_k} {\sigma'_k}
= \frac{1}{\lambda^2\sigma_k} \tanh \frac {s-\nu_k} {\sigma_k}\]

where $\kappa,\lambda$ do not depend on the subscripts $k$. It's easy to see that $\eta[f]$ and $\eta[g]$, too, are strictly increasing, with the same zero as $f$ and $g$, respectively.

We want to show that $\eta[g](x) \ge \eta[f](x)$ for all $x$. From this it will follow, as it did with $f$ and $g$, that the unique zero of $\eta[g]$ is no larger than that of $\eta[f]$.

Proof: we start by summarizing some of the above facts:

\[ \eta[f](s) \le \eta[f](r) = 0 = \eta[g](s) \le \eta[g](r) \]

For contradiction, suppose $\eta[g](t) < \eta[f](t)$ for some $t$. If $t\in [s,r]$, then

\[ 0 = \eta[g](s) \le \eta[g](t) < \eta[f](t) \le \eta[f](r) = 0 \]

which is a contradiction. For the rest of the proof, we can suppose $t < s$, since symmetry then takes care of the $t > r$ case. Let's only look at points $t$ and $s$. So far we have:

\[ f(t) < f(s) \le g(s) \]

\[ f(t) \le g(t) < g(s) \]

\[ \eta[g](t) < \eta[f](t) < \eta[f](s) \le 0 = \eta[g](s) \]

\subsection{Noising Methods}

So far, I can think of four reasonable noising methods. TODO: make this section into a table, with smileys replaced by colors: :D to green, :) to yellow, :( to red.

\subsubsection{Normal Approximation}

After each rating update, replace the posterior (which is the now product of one normal p.d.f. and one logistic p.d.f.) with a normal approximation. Then simply add Gaussian noise. This is the only method that does not accumulate a product of many logistic p.d.f.s, one for each recent performance.

Human-interpretable formulas: yes :D

Rating is a Markov state: yes :D

Monotonic in past performances: yes :D

Noising preserves MAP skill: yes :D

Saved history length: zero :D

Persistent outlier deweighting: immediate freeze-in, weights relative to rating at inception :)

Consistent change accelerates movement: no, slow convergence :(

\subsubsection{Constant Multiple on only the leading $\tau$}

The $\tau$ inside the $\tanh$ doesn't change. In other words, while the weight on each logistic factor is reduced over time, its variance is unchanged, so it never approaches the normal limit.

Human-interpretable formulas: yes :D

Rating is a Markov state: no :(

Monotonic in past performances: yes :D

Noising preserves MAP skill: yes :D

Saved history length: high :(

Persistent outlier deweighting: no freeze-in, weights relative to current rating :)

Consistent change accelerates movement: yes :D

\subsubsection{Constant Multiple on all $\tau$}

All of the $\tau$ change.

Human-interpretable formulas: yes :D

Rating is a Markov state: no :(

Monotonic in past performances: yes :D

Noising preserves MAP skill: no :(

Saved history length: low :)

Persistent outlier deweighting: outlierness is eventually forgotten, full weight restored :(

Consistent change accelerates movement: somewhat :)

\subsubsection{Fancy Method}

This is my favorite method. Regrettably, it fails the monotonicity criterion.

Human-interpretable formulas: yes :D

Rating is a Markov state: no :(

Monotonic in past performances: no :(

Noising preserves MAP skill: yes :D

Saved history length: low :)

Persistent outlier deweighting: gradual freeze-in, weights relative to neighbors in time :D

Consistent change accelerates movement: yes :D

\subsection{Robustness}

Imagine a player who performs very consistently over a long period of time, repeatedly achieving $p_i = 1000$ until convergence. Now, perhaps as a result of attending an intensive training camp in Petrozavodsk, their skill changes dramatically. From this point on, they consistently achieve $p_i = 3000$.

How does each rating system respond to the first such surprise occurrence? Elo-R treats the new result as a fluke, an outlier that ought to be ignored. The player gains 48 points; as a result of the parameters we set, this is the maximum possible for an experienced player as $p_i \rightarrow \infty$. In practice, ratings may change by more than 48, as the maximum depends on existing fluctuations in their history; here we're looking at the extreme example of a player with a history of always performing at exactly $p_i = 1000$.

\begin{center} \includegraphics[scale=0.5]{images/ResponsePlot.png} \end{center}

Had we tried to perform outlier reduction in a memoryless fashion, we would continue to increase the rating by 48 per match, oblivious to the possibility that the player truly did experience a sudden improvement. In Elo-R, the outlier status of a performance is treated as tentative. If later matches support the hypothesis of having improved, the rating will increase by an additional 63 points, followed by over 100 points in each of the third and following matches, as plotted by the blue curve above.

After six consecutive matches with $p_i = 3000$, the rating is 1875 and very unstable (even though $\sigma_i$ is unchanged!). The system is no longer sure which to trust: the extensive history at level 1000, or the smaller number of recent matches at level 3000. Depending on what comes next, the player's rating can very quickly fall toward 1000 or rise toward 3000. However, note that in either case, the change will not overshoot, say to 5000, unless enough new evidence is accumulated at that level. As the $p_i=3000$ streak continues, the seventh match on the blue curve jumps by a whopping 566 points. As the player's rating converges to 3000, the old $p_i = 1000$ data acquires outlier status, thus speeding convergence.

In contrast, while a system such as Codeforces does not compute $p_i$ values in quite in the same way, we can obtain a good approximation by removing outlier reduction from Elo-R, effectively treating the performances to be averaged as normal instead of logistic measurements. This makes the system effectively memoryless, since it turns out that each match simply moves the rating about 16\% closer to the new $p_i$ value, independent of the history. With this change, we obtain the orange curve, which jumps a whopping 320 points at the very first performance. Indeed, there is no limit: if you could find players whose ratings are extremely high, and beat them even once, your rating would take arbitrarily large leaps.

Note that this is not quite true of TopCoder, which incorporates a hack that caps the maximum rating change: if TopCoder's update formula demands too large a change, the cap kicks in. In contrast, Elo-R's cap is a natural and smooth consequence of its update formula and is sensitive to whether a change is charting new territory, or merely confirming a plausible hypothesis. TopCoder does attempt to make the magnitude of its updates sensitive to the amount of fluctuation in a player's history, using a volatility measure, but this measure does not account for the direction of the changes, resulting in the non-monotonicity flaw mentioned above.

Notwithstanding arguments that a high rating ought to properly be earned over multiple matches rather than a single fluke, the other danger is that these observations also hold in reverse: one bad day on Codeforces can seriously damage one's rating and negate several rounds of steady progress. By using heavy-tailed logistic distributions everywhere, Elo-R understands that unusually high or low performances do occasionally occur, and one round in isolation is never a reliable signal.

Interestingly, despite the slow start, the blue curve ultimately converges faster than the orange one. Since Elo-R uses its memory to dynamically adapt its view of potential outliers, it overtakes the orange curve as soon as new evidence outweighs the old hypothesis!

\subsection{Numerical analysis}

The ratings accumulate $O(\epsilon)$ numerical error per match, and likely a lot less in the long run due to statistical averaging...
\section{Experiments}
\label{sec:experiments}
In this section, we compare various rating systems on real-world datasets, mined from several sources that will be described in \Cref{sec:datasets}. The metrics are runtime and predictive accuracy, as described in \Cref{sec:metrics}. Implementations of all rating systems, dataset mining, and additional processing used in our experiments can be found at {\tt\url{https://github.com/EbTech/Elo-MMR}}.

We compare Elo-MM$\chi$ and Elo-MMR($\rho$) against the industry-tested rating systems of Codeforces and Topcoder. For a fairer comparison, we hand-coded efficient versions of all four algorithms in the safe subset of Rust, parellelized using the Rayon crate; as such, the Rust compiler verifies that they contain no data races~\cite{stone2017rayon}. Our implementation of Elo-MMR($\rho$) makes use of the optimizations in \Cref{sec:runtime}, bounding both the number of sampled opponents and the history length by 500. In addition, we test the improved TrueSkill algorithm of \cite{NS10}, basing our code on an open-source implementation of the same algorithm. The inherent seqentiality of its message-passing procedure prevented us from parallelizing it. All experiments were run on a 2.0 GHz 24-core Skylake machine with 24 GB of memory.

\paragraph{Hyperparameter search}
To ensure fair comparisons, we ran a separate grid search for each triple of algorithm, dataset, and metric, over all of the algorithm's hyperparameters. The hyperparameter set that performed best on the first 10\% of the dataset, was then used to test the algorithm on the remaining 90\% of the dataset. 

%We find that our rating performs slightly better than all competitors in terms of predictive power. In terms of computational time however, we show that Elo-MMR is up to an order of magnitude faster than Codeforces.

\subsection{Datasets}
\label{sec:datasets}

\begin{table}[t]
\begin{tabular}{l|l|l}
\hline
\textbf{Dataset} & \textbf{\# contests} & \textbf{avg. \# participants / contest} \\ \hline
Codeforces       & 1087                & 2999                                     \\ %\hline
Topcoder         & 2023                & 403                                   \\ %\hline
Reddit           & 1000                & 20                                       \\
%\hline
Synthetic        & 50                  & 2500     \\ \hline
\end{tabular}
    \caption{Summary of test datasets.}
    \label{tab:dataset-summary}
    \vspace{-1.2em}
\end{table}

Due to the scarcity of public domain datasets for rating systems, we mined three datasets to analyze the effectiveness of our system. The datasets were mined using data from each source website's inception up to October 9th, 2020. We also created a synthetic dataset to test our system's performance when the data generating process matches our theoretical model. Summary statistics of the datasets are presented in \Cref{tab:dataset-summary}.

\paragraph{Codeforces contest history}
This dataset contains the current entire history of rated contests ever run on codeforces.com, the dominant platform for online programming competitions. The Codeforces platform has over 850K users, over 300K of whom are rated, and has hosted over 1000 contests to date. Each contest has a couple thousand participants on average. A typical contest takes 2 to 3 hours and contains 5 to 8 problems. Players are ranked by total points, with more points typically awarded for tougher problems and for early solves. They may also attempt to ``hack'' one another's submissions for bonus points, identifying test cases that break their solutions. %The sheer number of highly motivated participants in these competitions, as well as their very accessible data API, made it the top choice for our explorations.
\looseness=-1

\paragraph{Topcoder contest history}
This dataset contains the current entire history of algorithm contests ever run on the topcoder.com. Topcoder is a predecessor to Codeforces, with over 1.4 million total users and a long history as a pioneering platform for programming contests. It hosts a variety of contest types, including over 2000 algorithm contests to date. The scoring system is similar to Codeforces, but its rounds are shorter: typically 75 minutes with 3 problems.

\paragraph{SubredditSimulator threads}
This dataset contains data scraped from the current top-1000 most upvoted threads on the website {\tt\url{reddit.com/r/SubredditSimulator/}}. Reddit is a social news aggregation website with over 300 million users. The site itself is broken down into sub-sites called subreddits. Users then post and comment to the subreddits, where the posts and comments receive votes from other users. In the subreddit SubredditSimulator, users are language generation bots trained on text from other subreddits. Automated posts are made by these bots to SubredditSimulator every 3 minutes, and real users of Reddit vote on the best bot. Each post (and its associated comments) can thus be interpreted as a round of competition between the bots who commented. 

\paragraph{Synthetic data}
This dataset contains 10K players, with skills and performances generated according to the Gaussian generative model in \Cref{sec:bayes_model}. Players' initial skills are drawn i.i.d. with mean $1500$ and variance $350^2$. Players compete in all rounds, and are ranked according to independent performances with variance $200^2$. Between rounds, we add i.i.d. Gaussian increments with variance $35^2$ to each of their skills.
% Uh is this the logistic or the Gaussian model??????

\subsection{Evaluation metrics}
\label{sec:metrics}
To compare the different algorithms, we define two measures of predictive accuracy. Each metric will be defined on individual contestants in each round, and then averaged:
\[\mathrm{\bf aggregate(metric)} := \frac{\sum_t \sum_{i\in\mathcal P_t} \mathrm{\bf metric}(i,t)}{\sum_t |\mathcal P_t|}.\]

\paragraph{Pair inversion metric~\cite{HMG06}}
Our first metric computes the fraction of opponents against whom our ratings predict the correct pairwise result, defined as the higher-rated player either winning or tying: 
\[\mathrm{\bf pair\_inversion}(i,t) := \frac{\text{\# correctly predicted matchups}}{|\mathcal P_t|-1} \times 100\%.\]
This metric was used in the original evaluation of TrueSkill~\cite{HMG06}.

\paragraph{Rank deviation}
Our second metric compares the rankings with the total ordering that would be obtained by sorting players according to their prior rating. The penalty is proportional to how much these ranks differ for player $i$:
\[\mathrm{\bf rank\_deviation}(i,t) := \frac{|\text{actual rank} - \text{predicted rank}|}{|\mathcal P_t|-1} \times 100\%.\]
In the event of ties, among the ranks within the tied range, we use the one that comes closest to the rating-based prediction.

% \paragraph{Entropy-based metric}
% For this metric, we evaluate the interpretability of the system ratings. As specified in \Cref{sec:bayes_model}, we assume player performances follow a Bradley-Terry model\paul{add BT and thurstone to this section}. In particular, we assume the probability of participants $i$ beating $j$ in a round $R$ is predicted by the simple formula \[\Pr[i \succ j] = \frac{1}{1 + 10^{\frac{\mu_i - \mu_j}{400}}}.\]
% As previously stated, this formula is assumed by the classic Elo rating system as well as the Codeforces rating system~\cite{...}, with the main benefit being that players can easily interpret the meaning of their ratings. To measure the interpretability, we measure the distance between the win distribution implied by the rating system and the actual win distribution. One way to do this is to measure the cross-entropy (which is equal to the KL-divergence up to an additive constant) via the follow formula:
% \[\mathrm{entropy} = -\frac{1}{\text{\# total pairs}} \sum_{\substack{i,j \in R \\ i \succ j}} \log \frac{1}{1 + 10^{\frac{\mu_i - \mu_j}{400}}}.\]

\subsection{Empirical results}
\begin{table*}
\begin{tabular}{l|ll|ll|ll|ll|ll}
 \hline
\multirow{2}{*}{\textbf{Dataset}} &
  \multicolumn{2}{l|}{\textbf{Codeforces}} &
  \multicolumn{2}{l|}{\textbf{Topcoder}} &
  \multicolumn{2}{l|}{\textbf{TrueSkill}} &
  \multicolumn{2}{l|}{\textbf{Elo-MM$\boldsymbol\chi$}} & 
  \multicolumn{2}{l}{\textbf{Elo-MMR($\boldsymbol\rho$)}} \\ \cline{2-11}
&
  pair inv. &
  rank dev. &
  pair inv. &
  rank dev. &
  pair inv. &
  rank dev. &
  pair inv. &
  rank dev. &
  pair inv. &
  rank dev. \\ \hline
Codeforces & 78.3\% & 14.9\% & 78.5\% & 15.1\% & 61.7\% & 25.4\% & 78.5\% & 14.8\% & {\bf 78.6}\% & {\bf 14.7}\% \\ %\hline
Topcoder  & 72.6\%     & 18.5\%     & 72.3\% & 18.7\%  & 68.7\% & 20.9\% & 73.0\% & 18.3\% & {\bf 73.1}\% & {\bf 18.2}\% \\ %\hline
Reddit     & 61.5\%     & 27.3\%     & 61.4\% & 27.4\% & 61.5\% & {\bf 27.2}\% & 61.6\% & 27.3\% & {\bf 61.6\%} & 27.3\% \\ %\hline
Synthetic  & {\bf 81.7\%}     & 12.9\%     & {\bf 81.7}\% & {\bf 12.8}\% & 81.3\% & 13.1\% & {\bf 81.7}\% & {\bf 12.8}\% & {\bf 81.7\%} & {\bf 12.8\%} \\ \hline
\end{tabular}
\caption{Performance of each rating system on the pairwise inversion and rank deviation metrics. Bolded entries denote the best performances (highest pair inv. or lowest rank dev.) on each metric and dataset.}
\label{tbl:metric-results}
\vspace{-1.2em}
\end{table*}

\begin{table}
\begin{tabular}{l|lllll}
\hline
\textbf{Dataset} & \textbf{CF} & \textbf{TC} & \textbf{TS} & \textbf{Elo-MM$\boldsymbol\chi$} & \textbf{Elo-MMR($\boldsymbol\rho$)} \\ \hline
Codeforces & 212.9 & 72.5 & 67.2 & {\bf 31.4} & 35.4\\
Topcoder   & 9.60 & {\bf 4.25} & 16.8 & 7.00 & 7.52\\
Reddit     & 1.19  & 1.14 & {\bf 0.44} & 1.14 & 1.42 \\
Synthetic  & 3.26  & 1.00 & 2.93 & {\bf 0.81} & 0.85 \\ \hline
\end{tabular}
\caption{Total compute time over entire dataset, in seconds.}
\label{tbl:time-results}
\vspace{-1.2em}
\end{table}

Recall that Elo-MM$\chi$ has a Gaussian performance model, matching the modeling assumptions of Topcoder and TrueSkill. Elo-MMR($\rho$), on the other hand, has a logistic performance model, matching the modeling assumptions of Codeforces and Glicko. While $\rho$ was included in the hyperparameter search, in practice we found that all values between $0$ and $1$ produce very similar results.

To ensure that errors due to the unknown skills of new players don't dominate our metrics, we excluded players who had competed in less than 5 total contests. In most of the datasets, this reduced the performance of our method relative to the others, as our method seems to converge more accurately. Despite this, we see in \Cref{tbl:metric-results} that both versions of Elo-MMR outperform the other rating systems in both the pairwise inversion metric and the ranking deviation metric.
\looseness=-1

We highlight a few key observations. First, significant performance gains are observed on the Codeforces and Topcoder datasets, despite these platforms' rating systems having been designed specifically for their needs. Our gains are smallest on the synthetic dataset, for which all algorithms perform similarly. This might be in part due to the close correspondence between the generative process and the assumptions of these rating systems. Furthermore, the synthetic players compete in all rounds, enabling the system to converge to near-optimal ratings for every player. Finally, the improved TrueSkill performed well below our expectations, despite our best efforts to improve it. We suspect that the message-passing numerics break down in contests with a large number of individual participants. The difficulties persisted in all TrueSkill implementations that we tried, including on Microsoft's popular {\tt Infer.NET} framework~\cite{InferNET18}. To our knowledge, we are the first to present experiments with TrueSkill on contests where the number of participants is in the hundreds or thousands. In preliminary experiments, TrueSkill and Elo-MMR score about equally when the number of ranks is less than about 60.

Now, we turn our attention to \Cref{tbl:time-results}, which showcases the computational efficiency of Elo-MMR. On smaller datasets, it performs comparably to the Codeforces and Topcoder algorithms. However, the latter suffer from a quadratic time dependency on the number of contestants; as a result, Elo-MMR outperforms them by almost an order of magnitude on the larger Codeforces dataset.

Finally, in comparisons between the two Elo-MMR variants, we note that while Elo-MMR($\rho$) is more accurate, Elo-MM$\chi$ is always faster. This has to do with the skill drift modeling described in \Cref{sec:skill-drift}, as every update in Elo-MMR($\rho$) must process $O(\log\frac 1\epsilon)$ terms of a player's competition history.

\section{Conclusions}

This paper introduces the Elo-R rating system, which is in part a generalization of the two-player Glicko system, allowing an unbounded number of players. It assumes the players' performances, while potentially hard to quantify directly, can be ranked in a total order. As a natural consequence of some technical modeling assumptions, Elo-R is far more robust to atypical performances than any alternative known to the author.

Applications include many types of sports and video games, as well as programming contests, which presently rely on less rigorously derived models and hacks. The modeling assumptions are best suited to events where the players have minimal targeted interactions against one another, and instead compete individually to score better than rival players in an ongoing array of challenges. For instance, suppose we want to measure a person's skill in traversing obstacle courses, where the course design changes weekly. Completion times are only meaningful on a single course. However, if we treat each course as a match in Elo-R, it becomes possible to quantify and compare the skills of individuals, even if they have never completed the same course together.

% IMPORTANT: This conference is double blind, which (aside from anonymous authors) means that we cannot post links to our git or have acknowledgements.

\section*{Acknowledgements}
The authors are indebted to Daniel Sleator and Danica J. Sutherland for initial discussions that helped inspire this work, and to Nikita Gaevoy for the open-source improved TrueSkill upon which our implementation is based. Experiments in this paper are funded by a Google Cloud Research Grant. The second author is supported by a VMWare Fellowship and the Natural Sciences and Engineering Research Council of Canada.

\balance

%\appendix
% \section*{Appendix}
\begingroup
\def\thetheorem{\ref{lem:decrease}}
\begin{lemma}
If $f_i$ is continuously differentiable and log-concave, then the functions $l_i,d_i,v_i$ are continuous, strictly decreasing, and
\[l_i(p) < d_i(p) < v_i(p) \text{ for all }p.\]
\end{lemma}
\addtocounter{theorem}{-1}
\endgroup
\begin{proof}
Continuity of $F_i,f_i,f'_i$ implies that of $l_i,d_i,v_i$. It's known~\cite{concave} that log-concavity of $f_i$ implies log-concavity of both $F_i$ and $1-F_i$. As a result, $l_i$, $d_i$, and $v_i$ are derivatives of strictly concave functions; therefore, they are strictly decreasing. In particular, each of

\[v'_i(p) = \frac{f'_i(p)}{F_i(p)} - \frac{f_i(p)^2}{F_i(p)^2},\quad
l'_i(p) = \frac{-f'_i(p)}{1-F_i(p)} - \frac{f_i(p)^2}{(1-F_i(p))^2},\]

are negative for all $p$, so we conclude that

\begin{align*}
d_i(p) - v_i(p)
= \frac{f'_i(p)}{f_i(p)} - \frac{f_i(p)}{F_i(p)}
&= \frac{F_i(p)}{f_i(p)} v'_i(p)
< 0,
\\l_i(p) - d_i(p)
= -\frac{f'_i(p)}{f_i(p)} -\frac{f_i(p)}{1-F_i(p)}
&= \frac{1-F_i(p)}{f_i(p)} l'_i(p)
< 0.
\end{align*}

\end{proof}

\iffalse

\begingroup
\def\thetheorem{\ref{thm:uniq-max}}
\begin{theorem}
Suppose that for all $j$, $f_j$ is continuously differentiable and log-concave. Then the unique maximizer of $\Pr(P_i=p\mid E^L_i,E^W_i)$ is given by the unique zero of
\[Q_i(p) = \sum_{j \succ i} l_j(p) + \sum_{j \sim i} d_j(p) + \sum_{j \prec i} v_j(p).\]
\end{theorem}
\addtocounter{theorem}{-1}
\endgroup

\begin{proof}
First, we rank the players by their buckets according to $\floor{P_j/\epsilon}$, and take the limiting probabilities as $\epsilon\rightarrow 0$:
\begin{align*}
    \Pr(\floor{\frac{P_j}\epsilon} > \floor{\frac{p}\epsilon})
    &= \Pr(p_j \ge \epsilon\floor{\frac{p}\epsilon} + \epsilon)
    \\&= 1 - F_j(\epsilon\floor{\frac{p}\epsilon} + \epsilon)
    \rightarrow 1 - F_j(p),
    \\\Pr(\floor{\frac{P_j}\epsilon} < \floor{\frac{p}\epsilon})
    &= \Pr(p_j < \epsilon\floor{\frac{p}\epsilon})
    \\&= F_j(\epsilon\floor{\frac{p}\epsilon})
    \rightarrow F_j(p),
    \\\frac 1\epsilon \Pr(\floor{\frac{P_j}\epsilon} = \floor{\frac{p}\epsilon})
    &= \frac 1\epsilon \Pr(\epsilon\floor{\frac{p}\epsilon} \le P_j < \epsilon\floor{\frac{p}\epsilon} + \epsilon)
    \\&= \frac 1\epsilon\left( F_j(\epsilon\floor{\frac{p}\epsilon} + \epsilon) - F_j(\epsilon\floor{\frac{p}\epsilon}) \right)
    \rightarrow f_j(p).
\end{align*}

Let $L_{jp}^\epsilon$, $W_{jp}^\epsilon$, and $D_{jp}^\epsilon$ be shorthand for the events $\floor{\frac{P_j}\epsilon} > \floor{\frac{p}\epsilon}$, $\floor{\frac{P_j}\epsilon} < \floor{\frac{p}\epsilon}$, and $\floor{\frac{P_j}\epsilon} = \floor{\frac{p}\epsilon}$. respectively. These correspond to a player who performs at $p$ losing, winning, and drawing against $j$, respectively, when outcomes are determined by $\epsilon$-buckets. Then,
\begin{align*}
\Pr(E^W_i,E^L_i\mid P_i=p)
&= \lim_{\epsilon\rightarrow 0}
\prod_{j \succ i} \Pr(L_{jp}^\epsilon)
\prod_{j \prec i} \Pr(W_{jp}^\epsilon)
\prod_{j \sim i, j\ne i} \frac{\Pr(D_{jp}^\epsilon)}\epsilon
\\&= \prod_{j \succ i} (1 - F_j(p)) \prod_{j \prec i} F_j(p) \prod_{j \sim i, j\ne i} f_j(p),
\\\Pr(P_i=p \mid E^L_i,E^W_i)
&\propto f_i(p) \Pr(E^L_i,E^W_i\mid P_i=p)
\\&= \prod_{j \succ i} (1 - F_j(p)) \prod_{j \prec i} F_j(p) \prod_{j \sim i} f_j(p),
\\\ddp\ln \Pr(P_i=p \mid E^L_i,& E^W_i) = \sum_{j \succ i} l_j(p) + \sum_{j \prec i} v_j(p) + \sum_{j \sim i} d_j(p) = Q_i(p).
\end{align*}

Since \Cref{lem:decrease} tells us that $Q_i$ is strictly decreasing, it only remains to show that it has a zero. If the zero exists, it must be unique and it will be the unique maximum of $\Pr(P_i=p \mid E^L_i,E^W_i)$.

To start, we want to prove the existence of $p^*$ such that $Q_i(p^*) < 0$. Note that it's not possible to have $f'_j(p) \ge 0$ for all $p$, as in that case the density would integrate to either zero or infinity. Thus, for each $j$ such that $j\sim i$, we can choose $p_j$ such that $f'_j(p_j) < 0$, and so $d_j(p_j) < 0$. Let $\alpha = -\sum_{j\sim i} d_j(p_j) > 0$.

Let $n = |\{j:\,j \prec i\}|$. For each $j$ such that $j \prec i$, since $\lim_{p\rightarrow\infty}v_j(p) = 0/1 = 0$, we can choose $p_j$ such that $v_j(p_j) < \alpha/n$. Let $p^* = \max_{j\preceq i} p_j$. Then,
\[
\sum_{j \succ i} l_j(p^*) \le 0, \quad \sum_{j \sim i} d_j(p^*) \le -\alpha, \quad \sum_{j \prec i} v_j(p^*) < \alpha.
\]

Therefore,
\begin{align*}
Q_i(p^*)
&= \sum_{j \succ i} l_j(p^*) + \sum_{j \sim i} d_j(p^*) + \sum_{j \prec i} v_j(p^*)
\\&< 0 - \alpha + \alpha = 0.
\end{align*}

By a symmetric argument, there also exists some $q^*$ for which $Q_i(q^*) > 0$. By the intermediate value theorem with $Q_i$ continuous, there exists $p\in (q^*,p^*)$ such that $Q_i(p) = 0$, as desired.
\looseness=-1
\end{proof}
\fi

\bibliographystyle{ACM-Reference-Format}
\bibliography{EloR}

\end{document}
